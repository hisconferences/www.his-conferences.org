
\section{Introduction}

With the rapid development of society and economy, more and more people live a stressful life. Too much stress threatens
human's physical and psychological health~\cite{60kasl1984,61hammen2005}.
Especially for the youth group, the threat is prone to such bad consequence as
depression or even suicide due to their spiritual immaturity~\cite{59Statistic}.
Hence, psychologists and educators have paid great attention to the adolescent stress issue~\cite{56colten1991,57wills1995}.
Nevertheless, one big difficulty in reality is that most growing adolescents are not willing or hesitate to
express their feelings to the people, but rather turn to the virtual world for stress release.
Chatterbot (also known as chatbot or talkbot), as a virtual artificial conversation system, can function as a
useful channel for cathartic stress relief, as it allows the
conversational partner to finally get to say what s/he cannot say out loud in real life.
Recently, \cite{25liu2013} proposed a chatting robot PAL, which
can answer non-obstructive psychological domain-specific questions. It
collects numerous psychological Q\&A pairs from the Q\&A community into a local knowledge base,
and selects a suitable answer to match the user's psychological question, taking
personal information into consideration.
However, in most situations, when users tell about their stress,
they wouldn't only ask some psychological questions. Instead, they pour forth their woes sentence by sentence.
and what the stressful people need is not only the solutions for their problems,
but also the feeling of being listened, understood, and comforted.
Comparatively, psychological domain-specific question-answer based chatting robots are rigid and insufficient in understanding and calming
users' stressful emotions.

In this study, we aim to build a adolescent-oriented
intelligent chatting system \emph{TeenChat}, which can on one hand sense adolescents' stress throughout the whole conversation rather than based on a single question each time in~\cite{25liu2013}, and on the other hand interact like a virtual friend to guide the stressful adolescents to gradually pour out their bad feelings by listening and comforting, and further encourage the adolescents by delivering positive messages and answers to their problems. To our knowledge, this is the first chatting system in the literature, designed specifically for sensing and helping release adolescents' stress in such stress categories as study, self-cognition, inter-personal, and affection during the virtual conversation. Our 1-month user study demonstrates that \emph{TeenChat's} has achieves 78.34\% precision and 76.12\% recall rate in stress sensing and making the stressful users feel better.

The reminder of the paper is organized as follows.
We review related work in Section 2, and outline our \emph{TeenChat} framework in
Section 3. Two core components for stress sensing and response generation
are detailed in Section 4 and 5, respectively.
Results of our user study are analyzed in Section 6.
Section 7 concludes the paper and discusses future work.

